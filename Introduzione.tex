\section{Introduzione}
Oggi giorno a livello mondiale le grandi aziende stanno introducendo per esempio sistemi di immagazzinaggio completamente automatizzato (nessun collaboratore presente), investendo capitali e risparmiando in costi di manodopera aumentando così la produttività all'estremo (24h, 365 giorni). La massima produttività, l'efficienza, la qualità, la ripetibilità (minor margine di scarto nella produzione) ed eliminare i punti deboli della forza lavoro (es. pause, riposo giornaliero, vacanze, assenze, infortuni, ecc.) sono i concetti chiave della quarta rivoluzione industriale che le aziende vogliono integrare per massimizzare i profitti.
L’obbiettivo del progetto è quello di riflettere sui cambiamenti che la quarta rivoluzione industriale porterà, come le nuove innovazioni modificheranno le strutture interne delle aziende e come questi due aspetti influenzano il mercato del lavoro. Il progetto si basa sulla seguente domanda di ricerca: \textbf{l’integrazione delle innovazioni tecnologiche della quarta rivoluzione industriale nelle imprese cambierà il mercato del lavoro in Ticino?}
Questa domanda permette di sviluppare un discorso che esplora la sfera economica che è fortemente toccata dagli effetti della quarta rivoluzione industriale. Gli attori coinvolti sono principalmente le imprese e la forza lavoro. Le aziende stanno integrando sistemi di automazione e d’intelligenza artificiale, vantaggiose per la loro imprenditorialità. Queste innovazioni molto spesso vengono introdotte a discapito della manodopera, modificando di conseguenza il mercato del lavoro e influenzando il tasso di disoccupazione. Questo tema inevitabilmente suscita riflessioni rispetto alla composizione futura del mercato del lavoro, alla nascita del concetto di personalizzazione di massa (mass customization) e allo sviluppo di prodotti smart.
